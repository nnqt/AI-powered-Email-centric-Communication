% this package is designed by: thanhhungqb@gmail.com
% more information and update: 
%   https://github.com/thanhhungqb/thesis-template
\documentclass[12pt,a4paper,oneside]{book} % twoside for draft

%\usepackage{babel}
\usepackage[utf8]{vietnam}
%\usepackage{times}
%\usepackage{graphicx}

\usepackage{mathptmx}	% same Time New Roma
%\renewcommand{\rmdefault}{phv} % Arial
%\renewcommand{\sfdefault}{phv} % Arial

\usepackage{fancyhdr}
\usepackage{algorithm2e}

\usepackage{float}
\usepackage{graphicx}
\graphicspath{{Images/}}

\usepackage{xcolor}
\usepackage{listings}

\definecolor{mGreen}{rgb}{0,0.7,0}
\definecolor{mGray}{rgb}{0.5,0.5,0.5}
\definecolor{mPurple}{rgb}{0.5,0,0.35}
\definecolor{backgroundColour}{rgb}{0.97,0.97,0.97}

\lstdefinestyle{CStyle}{
	backgroundcolor=\color{backgroundColour},
	commentstyle=\color{mGreen},
	keywordstyle=\color{red},
	numberstyle=\tiny\color{mGray},
	stringstyle=\color{mPurple},
	basicstyle=\footnotesize,
	breakatwhitespace=false,
	breaklines=true,
	captionpos=b,
	keepspaces=true,
	numbers=left,
	numbersep=5pt,
	showspaces=false,
	showstringspaces=false,
	showtabs=false,
	tabsize=2,
	language=C
}

\usepackage{bkthesis}

% Enable clickable TOC/LOF/LOT and PDF bookmarks (black links).
\usepackage[colorlinks=true,linkcolor=black,citecolor=black,urlcolor=black]{hyperref}
\hypersetup{pdfstartview=FitH,pdfpagemode=UseOutlines}

% Number and show headings down to \subsection.
\setcounter{secnumdepth}{2}
\setcounter{tocdepth}{2}

%\csdeptname{KHOA ĐIỆN ĐIỆN TỬ}
%\crname{BÁO CÁO THỰC TẬP TỐT NGHIỆP}
% \crname{BÁO CÁO TIỂU LUẬN}
\title{Nền tảng quản lý trao đổi Email và mở rộng đa kênh ứng dụng AI}
\cstuname{SVTH: Ngô Nguyễn Quốc Thịnh}

% \csCouncil{Khoa học máy tính}
\csSupervise{ThS. Võ Thanh Hùng}
\csReviewer{TS. Nguyễn Quốc Minh}
\cttime{12/2025}

\thesislayout

\begin{document}
%-	Bìa cứng - màu xanh dương, chữ mạ vàng (xem mẫu đính kèm)
%-	Trang tên (tờ lót): chất liệu giấy, nội dung giống như bìa LV
%-	Ở gáy LV: in nhan đề LV (có thể in tóm tắt nếu nhan đề quá dài), size 15 – 17
%-	Phiếu Nhiệm vụ LV, chấm điểm Hướng dẫn & Phản biện (đã ký): nhận từ GVHD & GVPB sau khi bảo vệ (theo lịch hẹn).
%-	Lời cam đoan
%-	Lời cảm ơn/ Lời ngỏ
%-	Tóm tắt LV
%-	Mục lục
%-	Danh mục, bảng biểu, hình ảnh, ... (nếu có)
%-	Nội dung LV
%-	Danh mục TL tham khảo
%-	Phụ lục (nếu có)

\coverpage

\frontmatter

% add content here
%-	Lời cam đoan
\begin{declaration}
	Tôi xin cam đoan đề cương luận văn tốt nghiệp này là công trình nghiên cứu và hiện thực của tôi, được thực hiện dưới sự hướng dẫn của \textit{GVHD ThS. Võ Thanh Hùng}.
	Kết quả trình bày trong báo cáo là trung thực, không sao chép từ bất kỳ công trình nào đã được công bố trước đó.
	Mọi tài liệu, số liệu, hình vẽ, thuật toán và nội dung tham khảo được sử dụng trong luận văn đều đã được trích dẫn và liệt kê phù hợp trong phần Tài liệu tham khảo.
	Tôi xin hoàn toàn chịu trách nhiệm trước Nhà trường và Khoa về mọi vấn đề liên quan đến tính trung thực học thuật và các hành vi đạo văn.

	\begin{flushright}
		TP. Hồ Chí Minh, Tháng 12 năm 2025

		Tác giả
		\vspace{2cm}

			\textit{Ngô Nguyễn Quốc Thịnh}
	\end{flushright}
\end{declaration}

%-	Lời cảm ơn/ Lời ngỏ
\begin{acknowledgments}
	Trước hết, tôi xin gửi lời cảm ơn chân thành đến \textit{GVHD ThS. Võ Thanh Hùng} đã tận tình hướng dẫn, góp ý chuyên môn và định hướng phương pháp trong suốt quá trình thực hiện luận văn.
	Những phản hồi kịp thời về mặt kiến trúc hệ thống, cách tiếp cận hiện thực PoC và cách trình bày báo cáo đã giúp tôi hoàn thiện đề tài đúng trọng tâm.

	Tôi xin cảm ơn Quý Thầy/Cô Khoa Khoa học và Kỹ thuật Máy tính, Trường Đại học Bách Khoa -- ĐHQG TP.HCM đã giảng dạy và cung cấp nền tảng kiến thức để tôi có thể thực hiện đề tài.
	Ngoài ra, tôi xin cảm ơn các bạn bè và người thân đã luôn động viên, hỗ trợ trong thời gian triển khai và hoàn thiện báo cáo.
\end{acknowledgments}

%-	Tóm tắt LV
\begin{abstract}
	Trong bối cảnh doanh nghiệp sử dụng đồng thời nhiều kênh giao tiếp như Email và các ứng dụng nhắn tin tức thời, dữ liệu trao đổi thường bị phân mảnh và khó truy vết đầy đủ bối cảnh.
	Bên cạnh đó, khối lượng nội dung lớn trong các luồng Email (threads) gây ra hiện tượng quá tải thông tin, làm giảm hiệu suất xử lý và khả năng nắm bắt vấn đề chính.

	Đồ án này đề xuất và hiện thực một nền tảng quản lý trao đổi tập trung với Email là lõi (Email-centric), đồng thời tăng cường bằng AI để hỗ trợ tổng hợp nội dung và giảm tải nhận thức cho người dùng.
	Hệ thống được thiết kế theo hướng module hoá, sẵn sàng mở rộng đa kênh trong tương lai.
	Trong phạm vi Proof of Concept (PoC), tôi tập trung hiện thực các chức năng cốt lõi gồm: (i) đồng bộ Email theo cơ chế kích hoạt thủ công (FR-01, manual sync), (ii) hiển thị Inbox và Timeline theo Thread (FR-04), và (iii) tóm tắt luồng trao đổi bằng AI (FR-07) theo mô hình gọi đồng bộ để đơn giản hoá triển khai demo.

	Về công nghệ, hệ thống sử dụng Next.js (Frontend và Backend API), FastAPI cho dịch vụ AI, MongoDB cho lưu trữ dữ liệu bán cấu trúc, Redis cho cache/queue theo định hướng mở rộng, và tích hợp Gmail API cùng Google Gemini API cho xử lý ngôn ngữ tự nhiên.
	Kết quả PoC cho thấy hướng tiếp cận Email-centric kết hợp AI có tiềm năng cải thiện khả năng quản lý bối cảnh trao đổi, đồng thời mở ra nền tảng kiến trúc cho các tính năng như cập nhật real-time, gợi ý phản hồi thông minh và tích hợp các kênh nhắn tin trong các giai đoạn tiếp theo.
\end{abstract}	
	
\tableofcontents
%\listofsymbols
\listoftables
\listoffigures
%\listofalgorithms


\mainmatter

\fancyhead{}  % Clears all page headers and footers
%\rhead{\thepage}  % Sets the right side header to show the page number
%\lhead{}  % Clears the left side page header
%\fancyfoot[positions]{footer}
\renewcommand{\footrulewidth}{0.4pt}

\pagestyle{fancy}  % Finally, use the "fancy" page style to implement the FancyHdr headers

\input{Chapters/Chapter1-Overview.tex}
\chapter{Phân tích và Đặc tả Yêu cầu}
\label{sec:YeuCau}

Dựa trên mục tiêu xây dựng một nền tảng quản lý giao tiếp tập trung với Email là lõi (Email-centric), chương này trình bày bức tranh tổng thể về \textbf{yêu cầu hệ thống} dưới góc nhìn vừa kỹ thuật vừa nghiệp vụ. Trong phạm vi PoC (Proof of Concept), hệ thống được giả định vận hành theo mô hình \textit{single-tenant} (một người dùng hoặc một tài khoản tổ chức), ưu tiên kiểm chứng tính khả thi của kiến trúc và năng lực xử lý dữ liệu kết hợp AI, hơn là giải quyết các bài toán multi-tenant ở quy mô lớn.

Để đảm bảo tính cô đọng, chương này tổ chức lại yêu cầu thành hai nhóm chính: \textbf{yêu cầu chức năng} và \textbf{yêu cầu phi chức năng}, thay vì tách quá nhiều tiểu mục nhỏ. Trong từng nhóm, các yêu cầu sẽ được mô tả theo góc nhìn "module lõi" của kiến trúc (Email, Contact, Timeline, AI, Đa kênh và Hạ tầng).

\section{Yêu cầu chức năng (Functional Requirements)}
\label{subsec:FR}

Nhóm yêu cầu chức năng mô tả \textit{hệ thống cần làm gì} để hỗ trợ người dùng quản lý email và thông tin liên lạc một cách tập trung, có ngữ cảnh và được tăng cường bởi AI.

\subsubsection*{(1) Lõi quản lý Email và đồng bộ hai chiều}

Email là kênh giao tiếp bắt buộc và là nguồn dữ liệu chính của hệ thống.

\begin{itemize}
    \item \textbf{FR-01 – Đồng bộ Email gần thời gian thực (near real-time sync):}
    Hệ thống cần tích hợp với Gmail API và sử dụng cơ chế \textit{Push Notification/Webhook} thay vì \textit{polling} thủ công. Khi có email mới tới hộp thư gốc, một sự kiện sẽ được đẩy tới backend, từ đó ghi nhận vào cơ sở dữ liệu và hiển thị lên giao diện Timeline trong khoảng thời gian mục tiêu dưới 5 giây. Cơ chế này không chỉ giúp giảm độ trễ mà còn tiết kiệm tài nguyên mạng và chi phí API.

    \item \textbf{FR-02 – Soạn thảo và gửi Email trực tiếp:}
    Người dùng có thể soạn email ngay trong giao diện hệ thống với hỗ trợ \textit{Rich Text} (in đậm, in nghiêng, bullet, trích dẫn) và đính kèm tệp. Email gửi đi từ nền tảng phải đồng bộ ngược lại vào Gmail để đảm bảo lịch sử ở hai phía luôn nhất quán, tránh tình trạng "mỗi nơi một mảnh".

    \item \textbf{FR-03 – Quản lý trạng thái Email và phản ánh lại máy chủ:}
    Hệ thống cho phép đánh dấu \textit{đã đọc/chưa đọc}, \textit{lưu trữ} (archive) hoặc gắn nhãn (labels/tags) đối với từng email. Các thay đổi này được đồng bộ ngược (two-way sync) về Gmail, giúp người dùng có thể sử dụng luân phiên cả giao diện gốc lẫn nền tảng PoC mà không bị lệch trạng thái.
\end{itemize}

\subsubsection*{(2) Dòng thời gian giao tiếp (Communication Timeline)}

Timeline là nơi thể hiện giá trị "Email-centric nhưng có ngữ cảnh".

\begin{itemize}
    \item \textbf{FR-04 – Hiển thị Inbox và Timeline theo Thread:}
    Hệ thống cần cung cấp giao diện Inbox hiển thị danh sách các thread email, sắp xếp theo thời gian của tin nhắn gần nhất. Mỗi thread hiển thị thông tin tóm tắt (snippet, người gửi, thời gian). Khi người dùng chọn một thread, hệ thống hiển thị toàn bộ lịch sử trao đổi theo dạng timeline, kết hợp với kết quả phân tích AI (nếu có). Đây là tính năng cốt lõi giúp người dùng có cái nhìn tổng quan về từng cuộc hội thoại.

    \item \textbf{FR-05 – Cập nhật giao diện tức thời (real-time UI update):}
    Khi backend nhận Webhook từ Gmail hoặc khi một tác vụ AI hoàn tất tóm tắt, giao diện Timeline phải được cập nhật tự động qua WebSocket hoặc cơ chế push tương đương, giúp người dùng luôn thấy trạng thái mới nhất mà không phải F5 thủ công.
\end{itemize}

\subsubsection*{(3) Chức năng AI: tóm tắt, gợi ý phản hồi và hợp nhất hồ sơ}

Module AI vận hành như một \textit{AI microservice} độc lập, giao tiếp với backend qua HTTP/REST.

\begin{itemize}
    \item \textbf{FR-06 – Tự động khởi tạo, làm giàu và hợp nhất hồ sơ Contact (AI-assisted Contact Management):}
    Khi xuất hiện một địa chỉ email mới trong luồng dữ liệu, backend ghi nhận thông tin kỹ thuật cơ bản (email, thời gian, header), còn AI service chạy nền để suy luận thêm metadata: tên hiển thị, ngôn ngữ thường sử dụng, domain liên quan tới tổ chức nào. Ngoài ra, dựa trên lịch sử giao tiếp và các đặc trưng như domain, chữ ký, cách xưng hô, AI service có thể đề xuất gộp nhiều địa chỉ email khác nhau vào cùng một Contact duy nhất. Người dùng xem xét và chấp nhận/từ chối gợi ý, sau đó hệ thống thực hiện hợp nhất nền để quy về một hồ sơ.

    \item \textbf{FR-07 – Tóm tắt luồng trao đổi (Thread Summarization):}
    Đối với các chuỗi email dài (thread), hệ thống theo yêu cầu người dùng gửi nội dung thread tới AI Service để sinh ra một bản tóm tắt ngắn gọn (summary) kèm theo danh sách "Key Issues" và "Action Required". Bản tóm tắt này được lưu lại trong DB, gắn với thread tương ứng và hiển thị trực tiếp trên giao diện Timeline để người dùng nắm nhanh "bức tranh tổng thể".

    \item \textbf{FR-08 – Gợi ý phản hồi thông minh (Smart Reply Suggestion):}
    Khi người dùng mở một email gần nhất trong thread, AI Service dựa trên nội dung email hiện tại cộng với ngữ cảnh lịch sử để đề xuất 2–3 phương án trả lời (reply candidates). Các gợi ý này ở dạng có thể chỉnh sửa lại trước khi gửi, đảm bảo vai trò AI là hỗ trợ (augmented intelligence) chứ không thay thế hoàn toàn con người.
\end{itemize}

\subsubsection*{(4) Mở rộng đa kênh và kiến trúc adapter}

Mặc dù PoC chỉ tích hợp Email, kiến trúc cần sẵn sàng cho việc bổ sung các kênh IM.

\begin{itemize}
    \item \textbf{FR-09 – Định nghĩa interface chuẩn cho Message/Conversation:}
    Hệ thống phải xây dựng các interface trừu tượng (ví dụ: \texttt{Message}, \texttt{Conversation}) độc lập với nguồn kênh. Điều này cho phép về sau có thể thêm các adapter cho Zalo, Telegram, WhatsApp mà không phải viết lại logic lõi (core business logic). PoC có thể minh họa cấu trúc này bằng các stub hoặc module giả lập.
\end{itemize}

\section{Yêu cầu phi chức năng (Non-Functional Requirements)}
\label{subsec:NFR}

Yêu cầu phi chức năng mô tả \textit{hệ thống cần vận hành như thế nào} về mặt hiệu năng, bảo mật và khả năng mở rộng, để đảm bảo PoC không chỉ chạy được mà còn phản ánh đúng định hướng kiến trúc cho tương lai.

\subsubsection*{(1) Hiệu năng và trải nghiệm người dùng}

\begin{itemize}
    \item \textbf{Độ trễ đồng bộ Email:}
    Thời gian từ khi email tới hộp thư gốc đến khi xuất hiện trên Timeline được đặt mục tiêu dưới 5 giây trong điều kiện mạng ổn định, nhờ cơ chế Webhook và xử lý bất đồng bộ ở backend.

    \item \textbf{Xử lý AI bất đồng bộ:}
    Các tác vụ AI như tóm tắt và gợi ý phản hồi có thể kéo dài từ vài giây đến hàng chục giây. Do đó, toàn bộ pipeline AI phải được thực hiện theo mô hình async, sử dụng hàng đợi (Redis Queue) để không chặn luồng xử lý chính. Người dùng vẫn nhìn thấy email mới gần như ngay lập tức, trong khi kết quả AI được cập nhật bổ sung sau.
\end{itemize}

\subsubsection*{(2) Bảo mật và quản lý cấu hình}

\begin{itemize}
    \item \textbf{Quản lý thông tin nhạy cảm qua môi trường:}
    Các thông tin như API key của nhà cung cấp LLM (Google Gemini, OpenAI), thông tin OAuth Client của Gmail, chuỗi kết nối (connection string) tới MongoDB và Redis phải được cấu hình qua biến môi trường (.env), không được ghi cứng trong mã nguồn. Điều này vừa tuân thủ thực hành bảo mật tốt, vừa giúp PoC dễ dàng triển khai trên nhiều môi trường khác nhau.

    \item \textbf{Phân tách quyền truy cập (ở mức PoC):}
    Mặc dù PoC không đi sâu vào bài toán authentication/authorization phức tạp, kiến trúc phải sẵn sàng để gắn thêm lớp xác thực (ví dụ: OAuth2, JWT) ở các lớp API mà không cần thay đổi cấu trúc tổng thể.
\end{itemize}

\subsubsection*{(3) Khả năng mở rộng và triển khai}

\begin{itemize}
    \item \textbf{Containerization toàn bộ thành phần:}
    Frontend (Next.js), Backend (Next.js API Routes), AI Service (FastAPI), Database (MongoDB) và Redis đều được đóng gói thành các container độc lập, được điều phối bởi Docker Compose. Cách tiếp cận này giúp mô phỏng khá sát môi trường triển khai thực tế và tạo tiền đề để sau này chuyển sang các nền tảng container orchestration (như Kubernetes) nếu cần.

    \item \textbf{Mở rộng theo chiều ngang (horizontal scalability) ở mức kiến trúc:}
    Dù PoC không bắt buộc chạy ở quy mô lớn, các lựa chọn công nghệ (Node.js event-driven, FastAPI async, Redis Queue) đều hướng tới khả năng nhân bản (scale-out) trong tương lai: có thể tăng số replica của Backend hoặc AI Service mà không phải thay đổi code nhiều.
\end{itemize}

\section{Quyết định công nghệ và Chiến lược kiến trúc}
\label{subsec:GiaiPhapCongNghe}

Dựa trên yêu cầu ở trên, báo cáo đưa ra một số quyết định công nghệ và chiến lược kiến trúc trọng yếu.

\subsubsection*{(1) Backend hướng sự kiện với Node.js/Next.js}

Backend sử dụng \textbf{Next.js API Routes} trong môi trường Node.js. Bài toán là \textit{I/O-bound}: hệ thống phải tiếp nhận webhook từ Gmail, kết nối tới MongoDB/Redis, đồng thời duy trì kênh WebSocket tới nhiều client.

So với Java (Spring Boot) hay Go, Node.js cho tốc độ phát triển PoC nhanh hơn, cấu hình đơn giản, hệ sinh thái npm phong phú. So với các framework Python hướng web truyền thống, mô hình event loop của Node.js phù hợp hơn cho việc xử lý đồng thời nhiều kết nối mạng. Đồng thời, dùng TypeScript ở cả frontend và backend giúp chia sẻ kiểu dữ liệu, giảm lỗi tích hợp.

\subsubsection*{(2) MongoDB cho dữ liệu bán cấu trúc}

Dữ liệu email là bán cấu trúc (semi-structured), các trường header và metadata đa dạng. \textbf{MongoDB} cho phép lưu trữ trực tiếp các document JSON từ Gmail API cùng với kết quả phân tích AI (summary, key\_issues, sentiment,…) mà không cần schema cứng như RDBMS. Điều này giúp PoC linh hoạt thay đổi mô hình dữ liệu trong giai đoạn khám phá (exploration), đồng thời vẫn đảm bảo hiệu năng ghi/đọc ở mức chấp nhận được.

\subsubsection*{(3) AI Microservice với Python/FastAPI}

Module AI được tách ra thành \textbf{FastAPI service} viết bằng Python để tận dụng hệ sinh thái AI/ML. Việc tách microservice này giúp:

\begin{itemize}
    \item Tránh làm nghẽn luồng xử lý của backend khi AI phải gọi các API có độ trễ cao (Google Gemini, OpenAI).
    \item Dễ dàng thay đổi nhà cung cấp AI hoặc chuyển sang mô hình local (HuggingFace, Ollama) trong tương lai mà không ảnh hưởng tới backend.
\end{itemize}

\subsubsection*{(4) Hàng đợi và xử lý bất đồng bộ với Redis}

Redis được sử dụng như một \textbf{message queue} đơn giản để hiện thực mô hình Producer–Consumer giữa backend và AI service. Khi có email mới, backend chỉ lưu dữ liệu, đẩy một job vào Redis rồi trả về kết quả tối thiểu cho frontend. AI service đọc job từ hàng đợi, xử lý tóm tắt/gợi ý và cập nhật lại DB, sau đó kích hoạt thông báo real-time cho người dùng.

\subsubsection*{(5) Cách tiếp cận AI: Prompt Engineering và Contextualization}

Trong giai đoạn PoC, hệ thống áp dụng hướng tiếp cận \textit{Prompt Engineering} với output có cấu trúc (structured JSON) thay vì fine-tuning. Các prompt được thiết kế để AI trả về cả nội dung tóm tắt lẫn thông tin có cấu trúc (sentiment, key\_issues, action\_required,...) nhằm phục vụ cho các logic hiển thị hoặc cảnh báo sau này. Đồng thời, AI luôn được cung cấp ngữ cảnh ở cấp độ thread thay vì từng email đơn lẻ, giúp kết quả tóm tắt và gợi ý phản hồi sát với diễn biến thực tế của cuộc trao đổi.
\input{Chapters/Chapter3-AIAndLLM.tex}
\chapter{Thiết kế Hệ thống}
\label{sec:ThietKe}

\section{Kiến trúc tổng thể hệ thống}
\label{subsec:OverallArchitecture}

\subsection{Sơ đồ kiến trúc hệ thống (Overall Architecture Diagram)}

\begin{figure}[H]
    \centering
    \includegraphics[width=1.0\textwidth]{architechture_diagram.png}
    \caption{Kiến trúc tổng quan của hệ thống Email-centric}
    \label{fig:overall-architecture}
\end{figure}

Hình~\ref{fig:overall-architecture} minh hoạ kiến trúc tổng thể của nền tảng Email-centric ở mức high-level. Hệ thống hoạt động dựa trên mô hình \textbf{Single Entry Point} (Điểm truy cập duy nhất) thông qua một Load Balancer, điều phối lưu lượng đến các cụm dịch vụ (Service Clusters) phi trạng thái phía sau. Các thành phần chính bao gồm:

\begin{itemize}
    \item \textbf{Load Balancer / Reverse Proxy (Nginx):} 
    Đóng vai trò là cổng giao tiếp duy nhất giữa Client và hạ tầng bên trong. 
    Thành phần này chịu trách nhiệm:
    \begin{itemize}
        \item Điều hướng (Routing) lưu lượng dựa trên đường dẫn: các yêu cầu trang web (\texttt{/}) được chuyển đến \textit{Frontend Cluster}, trong khi các yêu cầu dữ liệu (\texttt{/api}) được chuyển đến \textit{Backend Cluster}.
        \item Cân bằng tải (Load Balancing) để phân phối đều request giữa các bản sao (replicas) của dịch vụ, ngăn chặn tình trạng quá tải cục bộ.
    \end{itemize}

    \item \textbf{Frontend Cluster (Next.js):} 
    Bao gồm các bản sao của ứng dụng Next.js chạy song song. 
    Việc container hóa và cluster hóa Frontend đảm bảo khả năng xử lý lượng lớn yêu cầu truy cập đồng thời, đặc biệt là các tác vụ Render phía máy chủ (Server-Side Rendering - SSR). 
    Khi thực hiện SSR, các node Frontend cũng gọi API thông qua Load Balancer để đảm bảo tính nhất quán trong việc định tuyến nội bộ.

    \item \textbf{Backend Cluster (Next.js API):} 
    Là trung tâm xử lý logic nghiệp vụ, được thiết kế hoàn toàn \textit{Stateless} (phi trạng thái). 
    Các node Backend không lưu trữ session người dùng cục bộ mà ủy quyền cho tầng dữ liệu. 
    Thiết kế này cho phép hệ thống tự do thêm/bớt số lượng node Backend tùy theo tải thực tế mà không gây gián đoạn dịch vụ.

    \item \textbf{AI Worker Cluster (FastAPI):} 
    Là tập hợp các microservice worker chuyên trách xử lý các tác vụ AI nặng (tóm tắt thread, gợi ý phản hồi). 
    Các worker này hoạt động theo cơ chế bất đồng bộ (Asynchronous): chúng không nhận request trực tiếp từ User mà "lắng nghe" và xử lý các tác vụ (jobs) từ hàng đợi trong Redis. 
    Cơ chế này giúp tách biệt tải của AI ra khỏi luồng xử lý chính của người dùng.

    \item \textbf{Data \& State Layer (Redis + MongoDB):}
    \begin{itemize}
        \item \textbf{Redis (Shared Cache \& Queue):} Redis lưu trữ Session chung cho toàn bộ các cluster (giải quyết bài toán Stateless), đồng thời hoạt động như một Message Broker để điều phối hàng đợi công việc giữa Backend và AI Service.
        \item \textbf{MongoDB:} Lưu trữ bền vững (Persistent Storage) cho dữ liệu người dùng, email threads và siêu dữ liệu (metadata).
    \end{itemize}
\end{itemize}

Kiến trúc này cũng thể hiện rõ sự tương tác với các dịch vụ bên ngoài (External Cloud Services) như Gmail API (đồng bộ dữ liệu) và Google Gemini API (xử lý ngôn ngữ tự nhiên), đảm bảo hệ thống lõi chỉ tập trung vào logic nghiệp vụ và điều phối.

Sơ đồ thể hiện rõ đường đi của dữ liệu: từ Gmail API vào Backend, được lưu trong MongoDB, sau đó được Backend cung cấp cho Frontend để hiển thị Inbox/Timeline. Khi người dùng yêu cầu tóm tắt, Backend lấy dữ liệu thread từ MongoDB, gửi sang AI Service gọi Gemini, nhận kết quả tóm tắt và lưu trở lại, rồi trả về cho Frontend.

\subsection{Biểu đồ Use Case}
\label{subsubsec:UseCase}

\begin{figure}[H]
    \centering
    \includegraphics[width=1.0\textwidth]{Images/use_case_diagram.png} 
    \caption{Biểu đồ Use Case tổng thể của hệ thống (bao gồm thiết kế mở rộng đa kênh)}
    \label{fig:usecase}
\end{figure}

Hình~\ref{fig:usecase} mô tả các use case của hệ thống, làm rõ phạm vi chức năng và sự tương tác giữa người dùng với các hệ thống bên ngoài.

Các tác nhân (Actors) trong hệ thống bao gồm:
\begin{itemize}
    \item \textbf{Người dùng (Primary Actor):} Là người trực tiếp sử dụng hệ thống để quản lý giao tiếp (ví dụ: nhân viên tuyển dụng, nhân viên hỗ trợ).
    \item \textbf{Gmail API (Secondary Actor):} Hệ thống bên ngoài đóng vai trò cung cấp dữ liệu email gốc và thực hiện lệnh gửi email thực tế.
    \item \textbf{Google Gemini API (Secondary Actor):} Hệ thống trí tuệ nhân tạo bên ngoài, cung cấp khả năng xử lý ngôn ngữ tự nhiên cho các tác vụ thông minh.
    \item \textbf{Instant Message App:} Các ứng dụng nhắn tin tức thời như Slack, Microsoft Teams, Facebook Messenger, được tích hợp để mở rộng kênh giao tiếp.
\end{itemize}

Các Use Case được tổ chức theo luồng nghiệp vụ như sau:

\textbf{1. Nhóm Quản lý và Xem thông tin (Information Retrieval):}
\begin{itemize}
    \item \textbf{UC01 -- Đồng bộ Email:} Tương tác với Gmail API để cập nhật dữ liệu mới nhất về hệ thống.
    \item \textbf{UC02 -- Xem Inbox \& Timeline:} Cho phép người dùng xem danh sách tổng quan các luồng trao đổi.
    \item \textbf{UC03 -- Xem chi tiết Thread:} Là chức năng mở rộng (\textit{extend}) từ việc xem Inbox, cho phép đi sâu vào nội dung cụ thể của từng cuộc hội thoại.
\end{itemize}

\textbf{2. Nhóm Tác vụ Email (Email Actions):}
\begin{itemize}
    \item \textbf{UC06 -- Soạn Email:} Cung cấp trình soạn thảo văn bản (Rich Text Editor).
    \item \textbf{UC07 -- Gửi Email:} Thực hiện đẩy email đi thông qua Gmail API. Trong thiết kế hệ thống, chức năng Soạn thảo bao hàm (\textit{include}) chức năng Gửi như một bước hoàn tất quy trình.
\end{itemize}

\textbf{3. Nhóm Tính năng AI (AI-Assisted Features):}
\begin{itemize}
    \item \textbf{UC04 -- Tóm tắt Thread:} Được kích hoạt từ giao diện xem chi tiết (quan hệ \textit{extend}). Hệ thống gửi ngữ cảnh sang Google Gemini API để tạo bản tóm tắt.
    \item \textbf{UC05 -- Gợi ý phản hồi:} AI phân tích ngữ cảnh để đề xuất câu trả lời. Use case này có quan hệ bao hàm (\textit{include}) với UC06, nghĩa là khi chọn một gợi ý phản hồi, hệ thống sẽ tự động chuyển nội dung đó vào trình soạn thảo để người dùng tiếp tục chỉnh sửa.
\end{itemize}

\textbf{4. Nhóm Tích hợp Đa kênh (Multi-channel Integration - Design):}
\begin{itemize}
    \item \textbf{UC09 -- Tính năng hỗ trợ IM:} Thiết kế module kết nối với các nền tảng như Zalo và Microsoft Teams, nhằm gom dữ liệu chat về cùng giao diện Timeline với Email.
\end{itemize}

\section{Thiết kế luồng tương tác theo yêu cầu chức năng}
\label{subsec:SequenceDesign}

Các biểu đồ sequence mô tả luồng tương tác chi tiết giữa Frontend, Backend, AI Service và các thành phần hạ tầng cho một số FR trọng tâm.

\subsection{FR-01 -- Đồng bộ Email }
\label{subsubsec:SeqFR01}

\begin{figure}[H]
    \centering
    \includegraphics[width=0.95\textwidth]{Images/sequence_diagram_fr1.png} 
    \caption{Biểu đồ tuần tự cho chức năng Đồng bộ Email qua Webhook (FR-01)}
    \label{fig:seq-fr01}
\end{figure}

Để đáp ứng yêu cầu đồng bộ gần thời gian thực (Near Real-time), hệ thống được thiết kế sử dụng cơ chế \textbf{Push Notification} thông qua Google Cloud Pub/Sub thay vì kỹ thuật Polling truyền thống. Hình~\ref{fig:seq-fr01} minh họa luồng dữ liệu khi có email mới:

\begin{enumerate}
    \item \textbf{Trigger Sự kiện:} Khi email mới đến hộp thư Gmail, Google Cloud Pub/Sub tự động gửi một thông báo (Webhook) chứa \texttt{historyId} đến điểm cuối công khai của hệ thống thông qua Load Balancer.
    
    \item \textbf{Xác nhận và Xử lý:} Backend nhận thông báo và phản hồi mã \texttt{200 OK} ngay lập tức để xác nhận đã nhận tin. Sau đó, một tiến trình xử lý ngầm (background process) được kích hoạt.
    
    \item \textbf{Đồng bộ Dữ liệu (Sync):} Backend sử dụng \texttt{historyId} để truy vấn \textit{Gmail API}, lấy danh sách các thay đổi (incremental changes) và tải về nội dung email chi tiết. Dữ liệu sau đó được lưu trữ vào \textit{MongoDB}.
    
    \item \textbf{Phát tán Sự kiện (Real-time update):} Sau khi lưu trữ thành công, Backend đẩy một sự kiện \texttt{EMAIL\_RECEIVED} vào \textit{Redis Pub/Sub}. Các kết nối WebSocket từ Frontend sẽ lắng nghe sự kiện này để cập nhật giao diện Timeline tức thời (FR-04) mà không cần người dùng tải lại trang.
\end{enumerate}

\subsection{FR-02 -- Soạn thảo và gửi Email trực tiếp}
\label{subsubsec:SeqFR02}

\begin{figure}[H]
    \centering
    \includegraphics[width=0.95\textwidth]{Images/sequence_diagram_fr2.png}
    \caption{Biểu đồ tuần tự cho chức năng Soạn thảo và gửi Email (FR-02)}
    \label{fig:seq-fr02}
\end{figure}

FR-02 cho phép người dùng soạn email trực tiếp trên giao diện hệ thống và gửi thông qua Gmail API. Thiết kế nhấn mạnh hai điểm: (i) Gmail vẫn là nguồn thực thi hành động gửi (source of truth cho việc gửi) và (ii) hệ thống lưu lại bản ghi gửi đi để đảm bảo Inbox/Timeline hiển thị nhất quán và có thể truy vấn nhanh.

Luồng xử lý được chia thành các phase chính như sau:
\begin{enumerate}
    \item \textbf{Phase 1 - Soạn thảo \& Gửi (Compose \& Send):} Người dùng nhập nội dung và bấm gửi. Frontend gửi request \texttt{POST /api/emails/send} đến Backend thông qua Load Balancer.
    \item \textbf{Phase 2 - Gửi qua Gmail API:} Backend xác thực phiên, chuyển đổi nội dung sang định dạng Gmail yêu cầu và gọi Gmail API để gửi email. Gmail trả về \texttt{messageId} và \texttt{threadId} (trường hợp reply sẽ gắn vào thread sẵn có).
    \item \textbf{Phase 3 - Đồng bộ và Lưu trữ:} Backend tải chi tiết message vừa gửi (hoặc dùng payload đã có) để chuẩn hoá và lưu vào MongoDB dưới dạng \textit{Message}, đồng thời cập nhật \textit{Thread} (ví dụ \texttt{lastMessageDate}, \texttt{snippet}) để phục vụ hiển thị Inbox/Timeline.
    \item \textbf{Phase 4 - Cập nhật giao diện:} Backend phát sự kiện cập nhật (ví dụ \texttt{MESSAGE\_SENT} hoặc \texttt{THREAD\_UPDATED}) qua Redis Pub/Sub để Frontend có thể cập nhật UI tức thời theo cơ chế ở FR-05.
\end{enumerate}

\subsection{FR-03 -- Quản lý trạng thái Email và phản ánh lại máy chủ}
\label{subsubsec:SeqFR03}

\begin{figure}[H]
    \centering
    \includegraphics[width=0.95\textwidth]{Images/sequence_diagram_fr3.png}
    \caption{Biểu đồ tuần tự cho chức năng Quản lý trạng thái Email (FR-03)}
    \label{fig:seq-fr03}
\end{figure}

FR-03 cho phép người dùng thao tác trạng thái email (đã đọc/chưa đọc, archive, labels) trên nền tảng, đồng thời phản ánh (two-way sync) thay đổi này về Gmail để đảm bảo nhất quán giữa hai phía.

Luồng xử lý gồm các phase:
\begin{enumerate}
    \item \textbf{Phase 1 - Người dùng thao tác trạng thái:} Frontend gửi request (ví dụ \texttt{PATCH /api/messages/:id/state}) kèm danh sách labels cần thêm/bớt.
    \item \textbf{Phase 2 - Cập nhật Gmail:} Backend gọi Gmail API (modify labels) để áp dụng thay đổi lên message tương ứng.
    \item \textbf{Phase 3 - Cập nhật MongoDB:} Sau khi Gmail xác nhận thành công, Backend cập nhật bản ghi \textit{Message.labelIds} trong MongoDB để UI truy vấn nhanh mà không cần gọi Gmail liên tục.
    \item \textbf{Phase 4 - Thông báo real-time:} Backend phát sự kiện (ví dụ \texttt{MESSAGE\_STATE\_CHANGED}) để Frontend cập nhật UI theo FR-05.
\end{enumerate}

\subsection{FR-04 -- Xem Inbox và Thread Timeline}
\label{subsubsec:SeqFR04}

\begin{figure}[H]
    \centering
    \includegraphics[width=0.9\textwidth]{Images/sequence_diagram_fr4.png} 
    \caption{Biểu đồ tuần tự cho chức năng Xem Inbox (FR-04)}
    \label{fig:seq-fr04}
\end{figure}

Chức năng hiển thị Inbox và Timeline được thực hiện theo mô hình \textit{Client-side Data Fetching}, tận dụng khả năng của Next.js và thư viện SWR để tối ưu trải nghiệm người dùng. Hình~\ref{fig:seq-fr04} mô tả chi tiết hai phase của quá trình này:

\begin{enumerate}
    \item \textbf{Phase 1 - Tải giao diện (Initial Load):} 
    Khi người dùng truy cập đường dẫn, Load Balancer điều hướng yêu cầu tới \textit{Frontend Cluster}. Frontend trả về khung ứng dụng (App Shell) và các tài nguyên tĩnh (JS/CSS) để trình duyệt hiển thị giao diện sơ khởi (Skeleton UI) ngay lập tức.
    
    \item \textbf{Phase 2 - Lấy dữ liệu (Data Fetching):} 
    Sau khi giao diện tải xong, trình duyệt (Client) tự động gửi yêu cầu API bất đồng bộ (\texttt{GET /api/threads}) thông qua Load Balancer tới \textit{Backend Cluster}. Backend thực hiện xác thực phiên làm việc, truy vấn dữ liệu từ \textit{MongoDB}, và trả về kết quả dưới dạng JSON để Frontend cập nhật danh sách email đầy đủ.
\end{enumerate}

Thiết kế này giúp giảm tải cho Frontend Server (không phải chờ dữ liệu từ DB mới render HTML) và tăng cảm giác phản hồi nhanh cho người dùng.

\subsection{FR-05 -- Cập nhật giao diện tức thời (Real-time UI update)}
\label{subsubsec:SeqFR05}

\begin{figure}[H]
    \centering
    \includegraphics[width=0.95\textwidth]{Images/sequence_diagram_fr5.png}
    \caption{Biểu đồ tuần tự cho cơ chế Cập nhật giao diện tức thời (FR-05)}
    \label{fig:seq-fr05}
\end{figure}

FR-05 mô tả cơ chế push cập nhật từ Backend tới Frontend nhằm loại bỏ thao tác tải lại trang thủ công. Thiết kế sử dụng WebSocket để giữ kết nối dài hạn với client, kết hợp Redis Pub/Sub để phát tán sự kiện giữa các node Backend trong môi trường cluster.

Các phase chính:
\begin{enumerate}
    \item \textbf{Phase 1 - Thiết lập kênh real-time:} Frontend mở kết nối WebSocket tới hệ thống. Backend đăng ký (subscribe) theo ngữ cảnh người dùng.
    \item \textbf{Phase 2 - Nhận sự kiện nội bộ:} Khi một luồng khác tạo ra cập nhật (ví dụ sync email FR-01, gửi email FR-02, đổi trạng thái FR-03, hoàn tất AI FR-07/FR-08), Backend phát sự kiện vào Redis Pub/Sub.
    \item \textbf{Phase 3 - Push xuống client:} Backend nhận sự kiện từ Redis và đẩy thông báo xuống WebSocket. Frontend cập nhật UI và/hoặc re-fetch endpoint liên quan (threads hoặc thread detail) để đồng bộ dữ liệu.
\end{enumerate}

\subsection{FR-06 -- Tự động tạo \& hợp nhất Contact (AI)}
\label{subsubsec:SeqFR06}

\begin{figure}[H]
    \centering
    \includegraphics[width=0.95\textwidth]{Images/sequence_diagram_fr6.png}
    \caption{Biểu đồ tuần tự cho chức năng AI hỗ trợ tạo và hợp nhất Contact (FR-06)}
    \label{fig:seq-fr06}
\end{figure}

FR-06 hướng tới tổ chức lại dữ liệu liên lạc: khi xuất hiện địa chỉ mới, hệ thống tạo Contact cơ bản và dùng AI để làm giàu thông tin, đồng thời đề xuất hợp nhất nhiều định danh về cùng một hồ sơ. Thiết kế áp dụng mô hình \textit{human-in-the-loop}: AI chỉ \textbf{đề xuất}, còn quyết định hợp nhất do người dùng xác nhận.

Luồng xử lý gồm các phase:
\begin{enumerate}
    \item \textbf{Phase 1 - Khởi tạo Contact:} Backend phát hiện địa chỉ mới từ luồng đồng bộ/gửi email và tạo bản ghi Contact ở trạng thái \texttt{NEW}.
    \item \textbf{Phase 2 - AI Enrichment:} Backend đẩy một job vào Redis Queue. AI Worker lấy job, truy xuất ngữ cảnh (các email liên quan) rồi gọi Gemini để suy luận metadata (tên hiển thị, tổ chức/domain, ngôn ngữ, chữ ký, v.v.).
    \item \textbf{Phase 3 - Propose merge:} AI Worker sinh danh sách ứng viên có thể hợp nhất (merge candidates) kèm lập luận và mức tin cậy, lưu đề xuất ở trạng thái \texttt{PENDING}.
    \item \textbf{Phase 4 - Xác nhận người dùng \& cập nhật UI:} Hệ thống push thông báo qua FR-05 để Frontend hiển thị gợi ý. Người dùng approve/reject; nếu approve, Backend thực hiện hợp nhất và phát sự kiện cập nhật.
\end{enumerate}

\subsection{FR-07 -- Tóm tắt Thread}
\label{subsubsec:SeqFR07}

\begin{figure}[H]
    \centering
    \includegraphics[width=1.0\textwidth]{Images/sequence_diagram_fr7.png} 
    \caption{Biểu đồ tuần tự xử lý Tóm tắt AI bất đồng bộ (FR-07)}
    \label{fig:seq-fr07}
\end{figure}

Việc gọi API của các mô hình ngôn ngữ lớn (LLM) như Google Gemini thường có độ trễ cao (từ vài giây đến hàng chục giây tùy độ dài văn bản) và giới hạn số lượng yêu cầu (Rate Limiting). Nếu xử lý đồng bộ (Synchronous), Backend sẽ bị chiếm dụng tài nguyên kết nối trong thời gian chờ đợi, dễ dẫn đến nghẽn cổ chai.

Do đó, chức năng này được thiết kế theo mô hình \textbf{Bất đồng bộ (Asynchronous)} sử dụng Redis làm hàng đợi trung gian. Hình~\ref{fig:seq-fr07} mô tả quy trình 3 phase:

\begin{enumerate}
    \item \textbf{Phase 1 - Tiếp nhận (Dispatch):} 
    Khi nhận yêu cầu từ người dùng, Backend Cluster chỉ thực hiện việc truy xuất nội dung thread từ \textit{MongoDB}, đóng gói thành một bản tin (Job) và đẩy vào hàng đợi \textit{Redis Queue}. Backend lập tức phản hồi mã \texttt{202 Accepted} cho người dùng. Giao diện Frontend chuyển sang trạng thái "Đang xử lý" (Loading state) mà không bị treo.
    
    \item \textbf{Phase 2 - Xử lý nền (Processing):} 
    Các node trong \textit{AI Worker Cluster} (FastAPI) hoạt động độc lập, liên tục "lắng nghe" hàng đợi. Khi có Job mới, Worker sẽ lấy ra xử lý, gửi ngữ cảnh tới \textit{Google Gemini API} để sinh tóm tắt, sau đó cập nhật trực tiếp kết quả vào \textit{MongoDB}. 
    Thiết kế này cho phép tách biệt hoàn toàn tải của AI khỏi luồng phục vụ người dùng chính.
    
    \item \textbf{Phase 3 - Cập nhật (Notification):} 
    Sau khi cập nhật DB thành công, Worker phát một sự kiện thông báo qua kênh Redis Pub/Sub. Backend nhận sự kiện này và đẩy thông báo xuống Frontend (thông qua WebSocket hoặc Frontend tự động kiểm tra lại) để hiển thị kết quả cuối cùng.
\end{enumerate}

\subsection{FR-08 -- Gợi ý phản hồi thông minh (Smart Reply Suggestion)}
\label{subsubsec:SeqFR08}

\begin{figure}[H]
    \centering
    \includegraphics[width=0.95\textwidth]{Images/sequence_diagram_fr8.png}
    \caption{Biểu đồ tuần tự cho chức năng Gợi ý phản hồi thông minh (FR-08)}
    \label{fig:seq-fr08}
\end{figure}

FR-08 đề xuất 2--3 phương án trả lời dựa trên email mới nhất và ngữ cảnh thread. Tương tự FR-07, luồng xử lý được thiết kế theo hướng bất đồng bộ để tránh làm nghẽn Backend khi gọi LLM.

Các phase chính:
\begin{enumerate}
    \item \textbf{Phase 1 - Tiếp nhận yêu cầu:} Frontend gửi request \texttt{POST /api/threads/:id/suggest-reply}. Backend truy xuất ngữ cảnh (email mới nhất + lịch sử trao đổi cần thiết) từ MongoDB.
    \item \textbf{Phase 2 - Xử lý AI nền:} Backend đóng gói job và đẩy vào Redis Queue. AI Worker lấy job, gọi Gemini để sinh danh sách reply candidates.
    \item \textbf{Phase 3 - Trả kết quả và cập nhật UI:} Kết quả được lưu tạm (ví dụ Redis cache theo \textit{TTL} hoặc lưu bền tuỳ thiết kế chi tiết), sau đó hệ thống phát sự kiện qua Redis Pub/Sub để Frontend nhận và hiển thị ngay theo cơ chế FR-05.
\end{enumerate}

\section{Thiết kế Cơ sở dữ liệu (Database Schema)}
\label{subsec:DBDesign}

\begin{figure}[H]
    \centering
    \includegraphics[width=0.85\textwidth]{Images/erd_diagram.png} 
    \caption{Lược đồ Cơ sở dữ liệu (ERD) với chiến lược Nhúng và Tham chiếu}
    \label{fig:db-diagram}
\end{figure}

Hệ thống sử dụng \textbf{MongoDB} làm cơ sở dữ liệu chính. Để mô tả cấu trúc lưu trữ bán cấu trúc (NoSQL), lược đồ Quan hệ Thực thể (ERD) trong Hình~\ref{fig:db-diagram} minh họa các tập thực thể (Collections) và chiến lược liên kết dữ liệu.

Mô hình bao gồm ba thành phần chính:

\begin{enumerate}
    \item \textbf{User (Collection):} 
    Lưu trữ hồ sơ người dùng và các token xác thực OAuth2. Quan hệ giữa \textit{User} và \textit{Thread} là quan hệ 1-N (Một người dùng sở hữu nhiều luồng email).

    \item \textbf{Thread (Collection) và chiến lược Embedding:} 
    \textit{Thread} đại diện cho một luồng hội thoại. 
    Để tối ưu hiệu năng đọc (Read Performance) cho tính năng hiển thị tóm tắt trên Timeline (FR-07), thực thể \textbf{Summary} (chứa kết quả AI) được thiết kế theo chiến lược \textbf{Embedding (Nhúng)}. 
    Nghĩa là dữ liệu tóm tắt được lưu trực tiếp bên trong document của Thread thay vì tách ra bảng riêng, giúp giảm thiểu chi phí truy vấn (No Join).

    \item \textbf{Message (Collection) và chiến lược Referencing:} 
    \textit{Message} lưu trữ nội dung chi tiết của từng email.
    Trái ngược với Summary, quan hệ giữa \textit{Thread} và \textit{Message} sử dụng chiến lược \textbf{Referencing (Tham chiếu)}. 
    Các Message được lưu trong collection riêng biệt và liên kết thông qua khóa ngoại \texttt{threadId}. 
    Quyết định này nhằm đảm bảo khả năng mở rộng (Scalability), tránh lỗi tràn bộ nhớ document (16MB limit của MongoDB) đối với các luồng email dài chứa nhiều nội dung HTML hoặc đính kèm.
\end{enumerate}

\begin{table}[H]
    \centering
    \caption{Mô tả field-level cho collection \textit{User}}
    \label{tab:db-user-fields}
    \begin{tabular}{|p{2.2cm}|p{2.2cm}|p{7.2cm}|p{2.0cm}|}
        \hline
        	\textbf{Field} & \textbf{Type} & \textbf{Ý nghĩa} & \textbf{Liên quan FR} \\
        \hline
        	\texttt{email} & String & Email người dùng; dùng để định danh/tra cứu nhanh (được đánh index). & FR-01--FR-08 \\
        \hline
        	\texttt{name} & String & Tên hiển thị của người dùng (phục vụ UI). & FR-04 \\
        \hline
        	\texttt{image} & String & Ảnh đại diện (phục vụ UI). & FR-04 \\
        \hline
        	\texttt{googleId} & String & Định danh duy nhất từ Google OAuth (unique), dùng để liên kết phiên đăng nhập và token. & FR-01--FR-08 \\
        \hline
        	\texttt{accessToken} & String & Token truy cập Gmail API, cần để gọi các thao tác đồng bộ/gửi/sửa trạng thái email. & FR-01--FR-03 \\
        \hline
        	\texttt{refreshToken} & String & Token làm mới để gia hạn \texttt{accessToken} khi hết hạn, đảm bảo hệ thống hoạt động lâu dài. & FR-01--FR-03 \\
        \hline
        	\texttt{createdAt} & Date & Thời điểm tạo bản ghi (tự sinh bởi \texttt{timestamps}). & FR-01--FR-08 \\
        \hline
        	\texttt{updatedAt} & Date & Thời điểm cập nhật gần nhất (tự sinh bởi \texttt{timestamps}). & FR-01--FR-08 \\
        \hline
    \end{tabular}
\end{table}

\begin{table}[H]
    \centering
    \caption{Mô tả field-level cho collection \textit{Thread} (bao gồm Summary nhúng)}
    \label{tab:db-thread-fields}
    \begin{tabular}{|p{4.2cm}|p{2.2cm}|p{5.2cm}|p{2.0cm}|}
        \hline
        	\textbf{Field} & \textbf{Type} & \textbf{Ý nghĩa} & \textbf{Liên quan FR} \\
        \hline
        	\texttt{id} & String & Gmail thread ID (unique + index). Dùng để đồng bộ và ánh xạ dữ liệu giữa Gmail và hệ thống. & FR-01, FR-04 \\
        \hline
        	\texttt{userId} & ObjectId & Tham chiếu về \texttt{User}. Cho phép mỗi người dùng có tập thread riêng. & FR-01, FR-04 \\
        \hline
        	\texttt{historyId} & String & Mốc incremental sync từ Gmail (dùng để lấy thay đổi thay vì tải toàn bộ). & FR-01 \\
        \hline
        	\texttt{snippet} & String & Đoạn trích ngắn giúp hiển thị nhanh trên Inbox. & FR-04 \\
        \hline
        	\texttt{lastMessageDate} & Date & Thời điểm email gần nhất trong thread; dùng để sắp xếp Inbox theo độ mới. & FR-04 \\
        \hline
        	\texttt{summary.text} & String & Nội dung tóm tắt do AI sinh; nhúng vào Thread để đọc nhanh trên UI. & FR-07 \\
        \hline
        	\texttt{summary.key\_issues} & String[] & Danh sách vấn đề chính được trích xuất từ thread (AI). & FR-07 \\
        \hline
        	\texttt{summary.action\_required} & String[] & Danh sách hành động cần thực hiện (AI). & FR-07 \\
        \hline
        	\texttt{createdAt} & Date & Thời điểm tạo bản ghi (tự sinh bởi \texttt{timestamps}). & FR-01--FR-08 \\
        \hline
        	\texttt{updatedAt} & Date & Thời điểm cập nhật gần nhất (tự sinh bởi \texttt{timestamps}). & FR-01--FR-08 \\
        \hline
    \end{tabular}
\end{table}

\begin{table}[H]
    \centering
    \caption{Mô tả field-level cho collection \textit{Message}}
    \label{tab:db-message-fields}
    \begin{tabular}{|p{2.2cm}|p{2.2cm}|p{7.2cm}|p{2.0cm}|}
        \hline
        	\textbf{Field} & \textbf{Type} & \textbf{Ý nghĩa} & \textbf{Liên quan FR} \\
        \hline
        	\texttt{id} & String & Gmail message ID (unique + index). Dùng để tránh trùng lặp khi đồng bộ. & FR-01, FR-04 \\
        \hline
        	\texttt{threadId} & ObjectId & Tham chiếu về \texttt{Thread}. Giúp truy vấn danh sách email theo thread. & FR-04 \\
        \hline
        	\texttt{userId} & ObjectId & Tham chiếu về \texttt{User}. Giúp phân tách dữ liệu theo người dùng. & FR-01, FR-04 \\
        \hline
        	\texttt{from} & String & Địa chỉ người gửi; dùng hiển thị và phục vụ phân tích ngữ cảnh. & FR-04, FR-06 \\
        \hline
        	\texttt{to} & String[] & Danh sách người nhận; dùng hiển thị và tái dựng ngữ cảnh hội thoại. & FR-04, FR-06 \\
        \hline
        	\texttt{subject} & String & Tiêu đề email; phục vụ hiển thị và định hướng nội dung. & FR-04 \\
        \hline
        	\texttt{body} & String & Nội dung email (plain text/HTML tuỳ cách trích xuất). Là nguồn chính cho AI tóm tắt/gợi ý reply. & FR-07, FR-08 \\
        \hline
        	\texttt{snippet} & String & Đoạn trích ngắn từ nội dung; hỗ trợ hiển thị nhanh. & FR-04 \\
        \hline
        	\texttt{date} & Date & Thời điểm gửi/nhận email; dùng để sắp xếp timeline theo thời gian. & FR-04 \\
        \hline
        	\texttt{labelIds} & String[] & Nhãn Gmail (ví dụ \texttt{INBOX}, \texttt{UNREAD}); là cơ sở để đồng bộ trạng thái đọc/lưu trữ. & FR-03 \\
        \hline
        	\texttt{createdAt} & Date & Thời điểm tạo bản ghi (tự sinh bởi \texttt{timestamps}). & FR-01--FR-08 \\
        \hline
        	\texttt{updatedAt} & Date & Thời điểm cập nhật gần nhất (tự sinh bởi \texttt{timestamps}). & FR-01--FR-08 \\
        \hline
    \end{tabular}
\end{table}
\chapter{Hiện thực và Thử nghiệm PoC}
\label{sec:HienThuc}

\section{Môi trường triển khai và cách chạy PoC}
\label{subsec:EnvAndRun}

PoC được triển khai dưới dạng monorepo với cấu trúc thư mục chính:

\begin{itemize}
    \item \texttt{apps/frontend}: ứng dụng Next.js phục vụ giao diện người dùng (Inbox, Thread Timeline, AI Summary).
    \item \texttt{apps/backend}: ứng dụng Next.js API đóng vai trò Backend, tích hợp Gmail API, MongoDB và AI Service.
    \item \texttt{apps/ai-service}: dịch vụ FastAPI tách riêng cho các tác vụ AI (tóm tắt thread, gợi ý phản hồi) sử dụng Google Gemini.
    \item \texttt{infra}: cấu hình Docker Compose cho MongoDB, Redis và các dịch vụ liên quan.
\end{itemize}

\subsubsection*{Công nghệ và phiên bản chính}

Trong quá trình hiện thực, báo cáo đã sử dụng các công nghệ sau:

\begin{itemize}
    \item \textbf{Frontend}: Next.js (React + TypeScript), TailwindCSS.
    \item \textbf{Backend}: Next.js API Routes (Node.js, TypeScript), NextAuth cho đăng nhập Google.
    \item \textbf{AI Service}: FastAPI (Python), thư viện Google Generative AI để gọi Google Gemini.
    \item \textbf{Cơ sở dữ liệu}: MongoDB cho lưu trữ bán cấu trúc (User, Thread, Message).
    \item \textbf{Hạ tầng phụ trợ}: Redis cho cache và nền tảng queue/pub-sub trong tương lai; Docker Compose để khởi tạo nhanh môi trường cục bộ.
\end{itemize}

\subsubsection*{Cấu hình biến môi trường}

Toàn bộ các thông tin nhạy cảm (API key, OAuth client secret, connection string) đều được cấu hình thông qua file \texttt{.env} riêng cho từng service, thay vì ghi cứng trong mã nguồn:
\begin{itemize}
    \item \texttt{apps/backend/.env}: chứa chuỗi kết nối MongoDB, thông tin OAuth2 của Google, URL tới AI Service.
    \item \texttt{apps/ai-service/.env}: chứa API key và tên model của Google Gemini.
\end{itemize}

\subsubsection*{Quy trình khởi động PoC}

Để phục vụ cho việc thử nghiệm, báo cáo xây dựng một số script npm tại thư mục gốc giúp đơn giản hoá quy trình khởi động. Các bước tổng quát như sau:
\begin{enumerate}
    \item Cài đặt dependencies Node.js ở cấp độ monorepo:
    \begin{itemize}
        \item \texttt{npm install}
    \end{itemize}
    \item Chuẩn bị môi trường Python cho AI Service:
    \begin{itemize}
        \item \texttt{npm run dev:setup:ai} -- tạo virtualenv và cài đặt các thư viện cần thiết theo \texttt{requirements.txt}.
    \end{itemize}
    \item Khởi động hạ tầng dữ liệu (MongoDB, Redis) qua Docker Compose:
    \begin{itemize}
        \item \texttt{npm run dev:db}
    \end{itemize}
    \item Chạy đồng thời Backend, Frontend và AI Service trong môi trường phát triển:
    \begin{itemize}
        \item \texttt{npm run start:all}
    \end{itemize}
\end{enumerate}

Sau khi các dịch vụ khởi động thành công, có thể kiểm tra nhanh qua các endpoint sức khoẻ (health check):

\begin{itemize}
    \item Backend: \texttt{http://localhost:4000/api/health}.
    \item AI Service: \texttt{http://localhost:5000/}.
\end{itemize}

Giao diện người dùng (Frontend) được truy cập qua địa chỉ \texttt{http://localhost:3000}, nơi người dùng có thể đăng nhập bằng tài khoản Google, bấm nút đồng bộ email, xem Inbox/Thread và yêu cầu tóm tắt nội dung cuộc trao đổi.

\section{Tổng kết các chức năng đã hiện thực}
\label{subsec:FRSummary}

Bảng~\ref{tab:fr-impl-summary} tóm tắt mức độ hiện thực hoá các yêu cầu chức năng (FR) trong phạm vi PoC, đối chiếu với thiết kế ở các chương trước.

\begin{table}[H]
    \centering
    \caption{Tổng kết mức độ hiện thực các yêu cầu chức năng trong PoC}
    \label{tab:fr-impl-summary}
    \begin{tabular}{|p{1.5cm}|p{6cm}|p{3cm}|p{3cm}|}
        \hline
        & \textbf{Mô tả tóm tắt} & \textbf{Trạng thái trong PoC}  \\
        \hline
        FR-01 & Đồng bộ Email gần thời gian thực từ Gmail về hệ thống & Đã hiện thực một phần \\
        \hline
        FR-02 & Soạn thảo và gửi Email trực tiếp từ nền tảng & Chưa hiện thực \\
        \hline
        FR-03 & Quản lý trạng thái Email (đã đọc/chưa đọc, archive, labels) & Chưa hiện thực \\
        \hline
        FR-04 & Xem Inbox và Thread Timeline theo từng người dùng & Đã hiện thực \\
        \hline
        FR-05 & Cập nhật giao diện real-time khi có email mới hoặc kết quả AI & Chưa hiện thực \\
        \hline
        FR-06 & Tự động khởi tạo, làm giàu và hợp nhất hồ sơ Contact bằng AI & Chưa hiện thực \\
        \hline
        FR-07 & Tóm tắt luồng trao đổi (Thread Summarization) bằng AI & Đã hiện thực \\
        \hline
        FR-08 & Gợi ý phản hồi thông minh (Smart Reply Suggestion) & Chưa hiện thực \\
        \hline
        FR-09 & Kiến trúc mở rộng đa kênh (Multi-channel Adapter) & Chưa hiện thực \\
        \hline
    \end{tabular}
\end{table}
\chapter{Tổng kết}
\label{sec:KetLuan}

Chương này tổng kết toàn bộ kết quả thực hiện trong giai đoạn 1 (14 tuần), đồng thời đề xuất kế hoạch cho giai đoạn 2 (14 tuần) theo định hướng mở rộng đa kênh (multi-channel). Các kế hoạch được trình bày dưới dạng biểu đồ Gantt (PNG sẽ được chèn sau), trong đó mã Mermaid được lưu riêng để thuận tiện export.

\section{Tổng quan giai đoạn 1 (14 tuần)}

Trong giai đoạn 1, kỳ vọng ban đầu của đồ án tương đối lớn: vừa phải khảo sát thị trường, nghiên cứu nền tảng AI/LLM và công nghệ web, vừa thiết kế hệ thống chi tiết và triển khai PoC bao phủ nhiều yêu cầu chức năng (FR).


    \textbf{Kế hoạch ban đầu (Planned).} Trong 14 tuần, dự kiến dành khoảng \textbf{30\% thời gian cho triển khai PoC}, tương đương \textbf{4 tuần}. Phần còn lại dành cho nghiên cứu, thiết kế và viết báo cáo.


    \textbf{Thực tế (Actual).} Do quản lý thời gian chưa tốt và khối lượng phần lý thuyết/báo cáo lớn hơn dự kiến, thời gian còn lại cho triển khai PoC chỉ khoảng \textbf{10\%}, tương đương \textbf{1 tuần}

Vì vậy, PoC thực tế chỉ tập trung hoàn thiện một phần các luồng cốt lõi (FR-01/FR-04/FR-07) như đã tổng kết ở Chương~\ref{sec:HienThuc}.

\section{Kế hoạch ban đầu giai đoạn 1 (Planned)}

Kế hoạch ban đầu được tổ chức theo 5 nhóm công việc chính: (i) nghiên cứu đề tài và khảo sát giải pháp thị trường, (ii) nghiên cứu công nghệ và các phương pháp tiếp cận AI/LLM, microservices và web app, (iii) thiết kế hệ thống (kiến trúc và database), (iv) triển khai PoC, và (v) viết báo cáo.

Để phù hợp với bố cục trang, biểu đồ Gantt được chia thành 2 phần: Tuần 1--7 và Tuần 8--14.

\subsubsection*{Gantt (Planned) -- Tuần 1--7}
\begin{figure}[H]
    \centering
    \includegraphics[width=0.98\textwidth]{Images/gantt_gd1_planned_w1_7.png}
    \caption{Biểu đồ Gantt giai đoạn 1 (Planned) -- Tuần 1--7}
    \label{fig:gantt-gd1-planned-w1-7}
\end{figure}

\subsubsection*{Gantt (Planned) -- Tuần 8--14}
\begin{figure}[H]
    \centering
    \includegraphics[width=0.98\textwidth]{Images/gantt_gd1_planned_w8_14.png}
    \caption{Biểu đồ Gantt giai đoạn 1 (Planned) -- Tuần 8--14}
    \label{fig:gantt-gd1-planned-w8-14}
\end{figure}

\section{Kết quả thực tế giai đoạn 1 (Actual) và khó khăn}

\subsection{Những gì đã thực hiện}
Các kết quả chính trong giai đoạn 1 có thể tổng hợp theo 5 nhóm công việc như sau:
\begin{itemize}
    \item \textbf{Nghiên cứu đề tài và khảo sát thị trường:} tổng hợp các hướng tiếp cận Email client/CRM hỗ trợ AI; rút ra các yêu cầu thực tiễn như đồng bộ dữ liệu, phân loại luồng hội thoại, và trợ lý tóm tắt/phản hồi.
    \item \textbf{Nghiên cứu công nghệ liên quan:} tổng hợp các phương pháp tiếp cận AI/LLM (Prompt Engineering, fine-tuning), đánh giá lựa chọn microservices, và lựa chọn tech stack web app phù hợp cho PoC.
    \item \textbf{Thiết kế hệ thống:} hoàn thiện thiết kế kiến trúc các module chính và thiết kế database (User--Thread--Message, embed Summary), cùng các biểu đồ tuần tự cho các FR trọng yếu.
    \item \textbf{Triển khai PoC:} hiện thực các luồng cốt lõi gồm đồng bộ email (FR-01) ở mức PoC, hiển thị Inbox/Thread Timeline (FR-04) và tóm tắt thread bằng AI (FR-07). Các FR còn lại chủ yếu dừng ở mức thiết kế/đề xuất.
    \item \textbf{Viết báo cáo:} hoàn thiện phần tổng quan, phân tích yêu cầu, nền tảng công nghệ, thiết kế hệ thống và hiện thực PoC.
\end{itemize}

\subsection{Khó khăn và nguyên nhân dẫn đến kế hoạch không như dự kiến}
Theo Grantt chart của giai đoạn 1, khó khăn chính không nằm ở một điểm cụ thể, mà nằm ở \textbf{quản lý phạm vi và phân bổ thời gian}:
\begin{itemize}
    \item \textbf{Kỳ vọng ban đầu quá rộng} so với năng lực thời gian của 14 tuần, trong khi nhiều hạng mục có độ bất định cao (Gmail integration, mô hình dữ liệu, và pipeline AI bất đồng bộ).
    \item \textbf{Thời lượng cho phần lý thuyết và viết báo cáo bị kéo dài}, làm thời gian triển khai PoC bị co từ 4 tuần (kế hoạch) xuống còn 1 tuần (thực tế).
    \item \textbf{Chi phí tích hợp (integration cost)} giữa nhiều thành phần (Next.js, OAuth, Gmail API, MongoDB, AI Service) khiến triển khai end-to-end cần nhiều buffer hơn dự kiến.
\end{itemize}

\subsubsection*{Gantt (Actual) -- Tuần 1--7}
\begin{figure}[H]
    \centering
    \includegraphics[width=0.98\textwidth]{Images/gantt_gd1_actual_w1_7.png}
    \caption{Biểu đồ Gantt giai đoạn 1 (Actual) -- Tuần 1--7}
    \label{fig:gantt-gd1-actual-w1-7}
\end{figure}

\subsubsection*{Gantt (Actual) -- Tuần 8--14}
\begin{figure}[H]
    \centering
    \includegraphics[width=0.98\textwidth]{Images/gantt_gd1_actual_w8_14.png}
    \caption{Biểu đồ Gantt giai đoạn 1 (Actual) -- Tuần 8--14}
    \label{fig:gantt-gd1-actual-w8-14}
\end{figure}

\section{Kế hoạch giai đoạn 2 (14 tuần)}

Giai đoạn 2 được lập kế hoạch theo hướng \textbf{cân bằng giữa hoàn thiện phạm vi FR và các hạng mục mở rộng/kiểm thử}. 

\subsubsection*{Gantt (Giai đoạn 2) -- Tuần 1--7 (50\%: hoàn tất FR-01..FR-09)}
\begin{figure}[H]
    \centering
    \includegraphics[width=0.98\textwidth]{Images/gantt_gd2_plan_w1_7.png}
    \caption{Biểu đồ Gantt giai đoạn 2 (Plan) -- Tuần 1--7 (50\% hoàn tất FR)}
    \label{fig:gantt-gd2-plan-w1-7}
\end{figure}

\subsubsection*{Gantt (Giai đoạn 2) -- Tuần 8--14 (50\%: mở rộng, kiểm thử, tổng kết)}
\begin{figure}[H]
    \centering
    \includegraphics[width=0.98\textwidth]{Images/gantt_gd2_plan_w8_14.png}
    \caption{Biểu đồ Gantt giai đoạn 2 (Plan) -- Tuần 8--14 (50\% future, testing, report)}
    \label{fig:gantt-gd2-plan-w8-14}
\end{figure}

\section{Rủi ro giai đoạn 2 (Risks)}

Các rủi ro chính trong giai đoạn 2 và hướng giảm thiểu bao gồm:
\begin{itemize}
    \item \textbf{Rủi ro mở rộng phạm vi do multi-channel (scope creep):} dễ phát sinh yêu cầu tích hợp thêm nhiều kênh khi chưa ổn định lõi email. \textit{Giảm thiểu:} khoá phạm vi theo milestone 14 tuần; trong giai đoạn 2 chỉ triển khai FR-09 ở mức adapter baseline, ưu tiên hoàn tất FR-01..FR-08.
    \item \textbf{Rủi ro giới hạn Gmail API (quota/rate limit) và lỗi đồng bộ:} có thể ảnh hưởng trải nghiệm nếu sync không ổn định. \textit{Giảm thiểu:} throttling/backoff, lưu historyId đúng cách, bổ sung cơ chế retry và logging.
    \item \textbf{Rủi ro chi phí/độ trễ LLM:} các tác vụ FR-06/FR-07/FR-08 có thể tốn thời gian và chi phí token. \textit{Giảm thiểu:} chạy bất đồng bộ (queue), cache kết quả theo TTL, giới hạn context và chuẩn hoá prompt.
    \item \textbf{Rủi ro bảo mật và quyền riêng tư:} dữ liệu email nhạy cảm và token OAuth cần được xử lý chặt chẽ. \textit{Giảm thiểu:} giới hạn log, quản lý secret qua env, phân quyền truy cập dữ liệu theo userId, và cân nhắc chính sách lưu trữ/xoá dữ liệu trong các giai đoạn tiếp theo.
\end{itemize}

%-	Danh mục TL tham khảo
%-	Phụ lục (nếu có)

% Use manual bibliography to avoid requiring BibTeX/Biber.
% If you later want auto bibliography, revert to \bibliographystyle + \bibliography.
\input{manually.bbl}

\end{document}
