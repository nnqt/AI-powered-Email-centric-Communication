\chapter{Tổng kết}
\label{sec:KetLuan}

Chương này tổng kết toàn bộ kết quả thực hiện trong giai đoạn 1 (14 tuần), đồng thời đề xuất kế hoạch cho giai đoạn 2 (14 tuần) theo định hướng mở rộng đa kênh (multi-channel). Các kế hoạch được trình bày dưới dạng biểu đồ Gantt (PNG sẽ được chèn sau), trong đó mã Mermaid được lưu riêng để thuận tiện export.

\section{Tổng quan giai đoạn 1 (14 tuần)}

Trong giai đoạn 1, kỳ vọng ban đầu của đồ án tương đối lớn: vừa phải khảo sát thị trường, nghiên cứu nền tảng AI/LLM và công nghệ web, vừa thiết kế hệ thống chi tiết và triển khai PoC bao phủ nhiều yêu cầu chức năng (FR).


    \textbf{Kế hoạch ban đầu (Planned).} Trong 14 tuần, dự kiến dành khoảng \textbf{30\% thời gian cho triển khai PoC}, tương đương \textbf{4 tuần}. Phần còn lại dành cho nghiên cứu, thiết kế và viết báo cáo.


    \textbf{Thực tế (Actual).} Do quản lý thời gian chưa tốt và khối lượng phần lý thuyết/báo cáo lớn hơn dự kiến, thời gian còn lại cho triển khai PoC chỉ khoảng \textbf{10\%}, tương đương \textbf{1 tuần}

Vì vậy, PoC thực tế chỉ tập trung hoàn thiện một phần các luồng cốt lõi (FR-01/FR-04/FR-07) như đã tổng kết ở Chương~\ref{sec:HienThuc}.

\section{Kế hoạch ban đầu giai đoạn 1 (Planned)}

Kế hoạch ban đầu được tổ chức theo 5 nhóm công việc chính: (i) nghiên cứu đề tài và khảo sát giải pháp thị trường, (ii) nghiên cứu công nghệ và các phương pháp tiếp cận AI/LLM, microservices và web app, (iii) thiết kế hệ thống (kiến trúc và database), (iv) triển khai PoC, và (v) viết báo cáo.

Để phù hợp với bố cục trang, biểu đồ Gantt được chia thành 2 phần: Tuần 1--7 và Tuần 8--14.

\subsubsection*{Gantt (Planned) -- Tuần 1--7}
\begin{figure}[H]
    \centering
    \includegraphics[width=0.98\textwidth]{Images/gantt_gd1_planned_w1_7.png}
    \caption{Biểu đồ Gantt giai đoạn 1 (Planned) -- Tuần 1--7}
    \label{fig:gantt-gd1-planned-w1-7}
\end{figure}

\subsubsection*{Gantt (Planned) -- Tuần 8--14}
\begin{figure}[H]
    \centering
    \includegraphics[width=0.98\textwidth]{Images/gantt_gd1_planned_w8_14.png}
    \caption{Biểu đồ Gantt giai đoạn 1 (Planned) -- Tuần 8--14}
    \label{fig:gantt-gd1-planned-w8-14}
\end{figure}

\section{Kết quả thực tế giai đoạn 1 (Actual) và khó khăn}

\subsection{Những gì đã thực hiện}
Các kết quả chính trong giai đoạn 1 có thể tổng hợp theo 5 nhóm công việc như sau:
\begin{itemize}
    \item \textbf{Nghiên cứu đề tài và khảo sát thị trường:} tổng hợp các hướng tiếp cận Email client/CRM hỗ trợ AI; rút ra các yêu cầu thực tiễn như đồng bộ dữ liệu, phân loại luồng hội thoại, và trợ lý tóm tắt/phản hồi.
    \item \textbf{Nghiên cứu công nghệ liên quan:} tổng hợp các phương pháp tiếp cận AI/LLM (Prompt Engineering, fine-tuning), đánh giá lựa chọn microservices, và lựa chọn tech stack web app phù hợp cho PoC.
    \item \textbf{Thiết kế hệ thống:} hoàn thiện thiết kế kiến trúc các module chính và thiết kế database (User--Thread--Message, embed Summary), cùng các biểu đồ tuần tự cho các FR trọng yếu.
    \item \textbf{Triển khai PoC:} hiện thực các luồng cốt lõi gồm đồng bộ email (FR-01) ở mức PoC, hiển thị Inbox/Thread Timeline (FR-04) và tóm tắt thread bằng AI (FR-07). Các FR còn lại chủ yếu dừng ở mức thiết kế/đề xuất.
    \item \textbf{Viết báo cáo:} hoàn thiện phần tổng quan, phân tích yêu cầu, nền tảng công nghệ, thiết kế hệ thống và hiện thực PoC.
\end{itemize}

\subsection{Khó khăn và nguyên nhân dẫn đến kế hoạch không như dự kiến}
Theo Grantt chart của giai đoạn 1, khó khăn chính không nằm ở một điểm cụ thể, mà nằm ở \textbf{quản lý phạm vi và phân bổ thời gian}:
\begin{itemize}
    \item \textbf{Kỳ vọng ban đầu quá rộng} so với năng lực thời gian của 14 tuần, trong khi nhiều hạng mục có độ bất định cao (Gmail integration, mô hình dữ liệu, và pipeline AI bất đồng bộ).
    \item \textbf{Thời lượng cho phần lý thuyết và viết báo cáo bị kéo dài}, làm thời gian triển khai PoC bị co từ 4 tuần (kế hoạch) xuống còn 1 tuần (thực tế).
    \item \textbf{Chi phí tích hợp (integration cost)} giữa nhiều thành phần (Next.js, OAuth, Gmail API, MongoDB, AI Service) khiến triển khai end-to-end cần nhiều buffer hơn dự kiến.
\end{itemize}

\subsubsection*{Gantt (Actual) -- Tuần 1--7}
\begin{figure}[H]
    \centering
    \includegraphics[width=0.98\textwidth]{Images/gantt_gd1_actual_w1_7.png}
    \caption{Biểu đồ Gantt giai đoạn 1 (Actual) -- Tuần 1--7}
    \label{fig:gantt-gd1-actual-w1-7}
\end{figure}

\subsubsection*{Gantt (Actual) -- Tuần 8--14}
\begin{figure}[H]
    \centering
    \includegraphics[width=0.98\textwidth]{Images/gantt_gd1_actual_w8_14.png}
    \caption{Biểu đồ Gantt giai đoạn 1 (Actual) -- Tuần 8--14}
    \label{fig:gantt-gd1-actual-w8-14}
\end{figure}

\section{Kế hoạch giai đoạn 2 (14 tuần)}

Giai đoạn 2 được lập kế hoạch theo hướng \textbf{cân bằng giữa hoàn thiện phạm vi FR và các hạng mục mở rộng/kiểm thử}. 

\subsubsection*{Gantt (Giai đoạn 2) -- Tuần 1--7 (50\%: hoàn tất FR-01..FR-09)}
\begin{figure}[H]
    \centering
    \includegraphics[width=0.98\textwidth]{Images/gantt_gd2_plan_w1_7.png}
    \caption{Biểu đồ Gantt giai đoạn 2 (Plan) -- Tuần 1--7 (50\% hoàn tất FR)}
    \label{fig:gantt-gd2-plan-w1-7}
\end{figure}

\subsubsection*{Gantt (Giai đoạn 2) -- Tuần 8--14 (50\%: mở rộng, kiểm thử, tổng kết)}
\begin{figure}[H]
    \centering
    \includegraphics[width=0.98\textwidth]{Images/gantt_gd2_plan_w8_14.png}
    \caption{Biểu đồ Gantt giai đoạn 2 (Plan) -- Tuần 8--14 (50\% future, testing, report)}
    \label{fig:gantt-gd2-plan-w8-14}
\end{figure}

\section{Rủi ro giai đoạn 2 (Risks)}

Các rủi ro chính trong giai đoạn 2 và hướng giảm thiểu bao gồm:
\begin{itemize}
    \item \textbf{Rủi ro mở rộng phạm vi do multi-channel (scope creep):} dễ phát sinh yêu cầu tích hợp thêm nhiều kênh khi chưa ổn định lõi email. \textit{Giảm thiểu:} khoá phạm vi theo milestone 14 tuần; trong giai đoạn 2 chỉ triển khai FR-09 ở mức adapter baseline, ưu tiên hoàn tất FR-01..FR-08.
    \item \textbf{Rủi ro giới hạn Gmail API (quota/rate limit) và lỗi đồng bộ:} có thể ảnh hưởng trải nghiệm nếu sync không ổn định. \textit{Giảm thiểu:} throttling/backoff, lưu historyId đúng cách, bổ sung cơ chế retry và logging.
    \item \textbf{Rủi ro chi phí/độ trễ LLM:} các tác vụ FR-06/FR-07/FR-08 có thể tốn thời gian và chi phí token. \textit{Giảm thiểu:} chạy bất đồng bộ (queue), cache kết quả theo TTL, giới hạn context và chuẩn hoá prompt.
    \item \textbf{Rủi ro bảo mật và quyền riêng tư:} dữ liệu email nhạy cảm và token OAuth cần được xử lý chặt chẽ. \textit{Giảm thiểu:} giới hạn log, quản lý secret qua env, phân quyền truy cập dữ liệu theo userId, và cân nhắc chính sách lưu trữ/xoá dữ liệu trong các giai đoạn tiếp theo.
\end{itemize}